\documentclass[12pt]{article}
\usepackage[utf8]{inputenc}
\usepackage{listings}
\usepackage{xcolor}

% Definición del lenguaje JavaScript para listings
\lstdefinelanguage{JavaScript}{
  keywords={break, case, catch, class, const, continue, debugger, default, delete, do, else, export, extends, false, finally, for, function, if, import, in, instanceof, new, null, return, super, switch, this, throw, true, try, typeof, var, void, while, with, let, static, yield},
  keywordstyle=\color{blue},
  ndkeywords={await, async},
  ndkeywordstyle=\color{blue},
  sensitive=true,
  comment=[l]{//},
  morecomment=[s]{/*}{*/},
  commentstyle=\color{gray},
  stringstyle=\color{red},
  morestring=[b]',
  morestring=[b]"
}

\lstset{
  basicstyle=\ttfamily\small,
  keywordstyle=\color{blue},
  stringstyle=\color{red},
  commentstyle=\color{gray},
  showstringspaces=false,
  frame=single,
  breaklines=true,
  literate=
  {á}{{\'a}}1
  {é}{{\'e}}1 
  {í}{{\'\i}}1 
  {ó}{{\'o}}1 
  {ú}{{\'u}}1 
  {ñ}{{\~n}}1
}

\begin{document}

\title{Características que hacen a JavaScript orientado a objetos}
\author{}
\date{}
\maketitle

% \section*{Respuesta}



% \section*{Modelos sin clases}

% Existen varios lenguajes que implementan POO sin recurrir a clases tradicionales, utilizando en su lugar \emph{prototipos} o \emph{mensajes}:

% \begin{itemize}
%   \item \textbf{JavaScript (pre-ES6):} Usa prototipos para herencia y creación de objetos, sin sintaxis de \lstinline|class|.
%   \item \textbf{Self:} Lenguaje puramente basado en prototipos, sin clases.
%   \item \textbf{Lua:} Se apoya en tablas y metatables para simular objetos y herencia.
%   \item \textbf{Io:} Sistema de objetos basado en envío de mensajes, sin clases.
% \end{itemize}

% \section*{Conclusión}

% Las \emph{clases} son solo un mecanismo de azúcar sintáctico para generar y organizar objetos. Un lenguaje es orientado a objetos si ofrece herramientas para crear y manipular objetos, independientemente de si utiliza clases o prototipos.


\section*{Introducción}

\subsection*{¿Es necesario tener clases para ser orientado a objetos?}

No es indispensable que un lenguaje disponga de \emph{clases} para ser considerado orientado a objetos. Lo fundamental en la Programación Orientada a Objetos (POO) son los \textbf{objetos} y los mecanismos que permiten:

\begin{enumerate}
  \item \textbf{Encapsulamiento:} Agrupar datos y comportamientos en una misma unidad.
  \item \textbf{Abstracción:} Ocultar detalles internos y exponer una interfaz limpia.
  \item \textbf{Herencia:} Compartir y reutilizar comportamiento entre objetos.
  \item \textbf{Polimorfismo:} Permitir que objetos distintos respondan al mismo mensaje de formas diferentes.
\end{enumerate}

JavaScript es un lenguaje orientado a objetos que, aunque basa su modelo en prototipos en lugar de clases tradicionales, cumple plenamente con los cuatro pilares fundamentales de la Programación Orientada a Objetos (POO):

\section*{JavaScript cumple cada uno de estos principios}

\section{Encapsulamiento}

El encapsulamiento agrupa datos y comportamientos en una única entidad: el objeto.

\subsection*{Ejemplo con objeto literal}

\begin{lstlisting}[language=JavaScript]
const coche = {
  marca: "Toyota",
  encender: function() {
    console.log("Coche encendido");
  }
};

coche.encender();  // Coche encendido
\end{lstlisting}

\subsection*{Ejemplo con sintaxis de clases (ES6)}

\begin{lstlisting}[language=JavaScript]
class Coche {
  constructor(marca) {
    this.marca = marca;
  }
  encender() {
    console.log(`${this.marca} encendido`);
  }
}
\end{lstlisting}

\section{Abstracción}

La abstracción oculta la complejidad interna y muestra solo lo necesario.

\subsection*{Campos privados (ES2022)}

\begin{lstlisting}[language=JavaScript]
class Cuenta {
  #saldo = 0; // propiedad privada

  depositar(cantidad) {
    if (cantidad > 0) this.#saldo += cantidad;
  }

  verSaldo() {
    return this.#saldo;
  }
}

const c = new Cuenta();
c.depositar(100);
console.log(c.verSaldo()); // 100
// c.#saldo -> Error: campo privado
\end{lstlisting}

\section{Herencia}

JavaScript implementa herencia prototípica; desde ES6 también soporta sintaxis de clases.

\subsection*{Herencia prototípica}

\begin{lstlisting}[language=JavaScript]
function Animal(nombre) {
  this.nombre = nombre;
}
Animal.prototype.hablar = function() {
  console.log(`${this.nombre} hace un sonido`);
};

function Perro(nombre) {
  Animal.call(this, nombre);
}
Perro.prototype = Object.create(Animal.prototype);
Perro.prototype.ladrar = function() {
  console.log(`${this.nombre} ladra`);
};

const firulais = new Perro("Firulais");
firulais.hablar();  // Firulais hace un sonido
firulais.ladrar();  // Firulais ladra
\end{lstlisting}

\subsection*{Herencia con clases (ES6)}

\begin{lstlisting}[language=JavaScript]
class Animal {
  constructor(nombre) {
    this.nombre = nombre;
  }
  hablar() {
    console.log(`${this.nombre} hace un sonido`);
  }
}

class Perro extends Animal {
  ladrar() {
    console.log(`${this.nombre} ladra`);
  }
}
\end{lstlisting}

\section{Polimorfismo}

El polimorfismo permite que distintos objetos respondan al mismo método de maneras diferentes.

\begin{lstlisting}[language=JavaScript]
class Animal {
  hablar() {
    console.log("Sonido genérico");
  }
}

class Gato extends Animal {
  hablar() {
    console.log("Miau");
  }
}

class Perro extends Animal {
  hablar() {
    console.log("Guau");
  }
}

const animales = [new Gato(), new Perro()];
animales.forEach(animal => animal.hablar());
// Miau
// Guau
\end{lstlisting}

\section*{Otras características relacionadas}

\begin{itemize}
  \item \textbf{Funciones de primera clase:} métodos son funciones que pueden pasarse como argumentos.
  \item \textbf{Objetos dinámicos:} se pueden modificar (agregar/quitar propiedades) en tiempo de ejecución.
  \item \textbf{Closures:} permiten emular encapsulamiento privado antes de los campos \texttt{\#private}.
\end{itemize}

\section*{Conclusión}

JavaScript cumple con los principios de la POO gracias a su modelo de objetos basado en prototipos y, desde ES6, con sintaxis de clases y campos privados. Aunque no usara clases originalmente, incorpora todos los pilares fundamentales de la programación orientada a objetos.

\end{document}
